\section{Présentation du jeu}
Le but du projet est de développer est un jeu de stratégie au tour par tour. Chaque joueur contrôle un peuple et doit gérer des unités sur une carte afin d'obtenir le plus de points possible après un certain nombre de tours de jeu. Détaillons-en les différentes spécificités.
\subsection{Les peuples}
Il existe trois peuples : les Elfs, les Orcs et les Nains, dont les caractéristiques influent sur les différentes stratégies de jeu. Un peuple ne peut être choisi que par un seul joueur et a des bonus/malus en fonction des conditions environnementales ou des actions effectuées par les unités. Par exemple, le coût d'un déplacement en forêt, pour un elf, est divisé par deux. Autre exemple, un orc qui tue une unité au combat gagne un point bonus permanent.

\subsection{La Carte du monde}
La carte du monde se compose de cases hexagonales dont il existe différents types : Désert, Montagne, Forêt et Plaine. Ces différents terrains interagissent avec les unités en fonction du peuple auquel elles appartiennent, permettant ainsi de développer plusieurs stratégies.

\subsection{Types de cartes}
Il existe 3 types de cartes :
\begin{itemize}
  \item Démo : 2 joueurs, 6 cases x 6 cases, 4 unités par joueur.
  \item Petite : 2 joueurs, 10 cases x 10 cases, 6 unités par joueur.
  \item Normale :  2 joueurs, 14 cases x 14 cases, 8 unités par joueur.
\end{itemize}
Le nombre de case est multiple de 4, puisque le terrain comporte autant de cases de chaque type disponible.
\subsection{Les Combats}
Pour qu'une unité attaque une autre unité, ils faut qu'elles se trouvent sur des cases adjacentes. Lorsqu'une unité attaque une case contenant plusieurs unités, la meilleure unité défensive est choisie. Chaque combat calcule les probabilités de perte d'une vie de l'attaquant selon la formule suivante :
\bigskip
\begin{minipage}[b]{1\linewidth}
\centering
\begin{equation*}
P = \frac{1 - (A - D)}{2} + 0,5
\end{equation*}
\medskip
\textit{Calcul de la probabilité pour l'attaquant de perdre une vie}
\end{minipage}%

\bigskip
\begin{itemize}
  \item \ensuremath{P}: probabilité de perte d'une vie de l'attaquant
  \item \ensuremath{A}: points d'attaque de l'unité offensive
  \item \ensuremath{D}: points de défense de l'unité attaquée.
\end{itemize}
\bigskip

Par exemple, prenons une unité ayant 3 points d'attaque déclenche un combat contre une autre unité ayant 4 points de défense, selon cette formule, il aura 62.5\% de chances de perdre une vie.\\

De plus, les points de vie rentrent en compte dans le calcul des probabilités : on pondère le nombre de points d'attaque par le pourcentage de points de vie restants. Une unité avec 4 points d'attaque de base et 50\% de sa vie se retrouvera, au final, avec 2 points d'attaque.\\\\
Enfin, lorsqu'un attaquant gagne le combat, il se déplace automatiquement sur la case du défenseur si elle est vide. Cela lui coûte un point de déplacement quel que soit le type de la case ou de l'unité.

\subsection{Le jeu}
\subsubsection{Début du jeu}
Chaque joueur commence la partie en sélectionnant son peuple (le choix du premier joueur est fait aléatoirement). Pour commencer, les unités du peuples sont placées sur la même case. Les joueurs jouent tour à tour sur le même ordinateur.
\subsubsection{Déroulement d'un tour}
Lorsqu'un joueur peut jouer, il peut choisir de déplacer ses unités ou d'attaquer une unité adverse s'il est à portée. Lorsqu'il a fini son tour, il clique sur le bouton Fin de tour, ce qui permet à l'autre de commencer son tour.
