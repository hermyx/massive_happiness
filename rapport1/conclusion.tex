\section{Conclusion}

L'étude préliminaire pour le jeu que nous allons développer nous a permis de nous donner une direction pour l'implémentation : un code générique, structuré autour des concepts de joueurs, d'unités etc ... Les diagrammes ne sont pas gravés dans la roche, et les étapes d'implémentation nous pousseront peut-être à les modifier, mais ils constituent une base saine pour développer.
Cette première étape ne prend pas en compte les éléments d'affichage et d'interface graphique (modèle MVC, implémentation WPF...).

La prochaine étape du projet sera l'implémentation concrète de l'application. Nous commencerons par implémenter le fonctionnement du jeu, les différentes méthodes des classes. Ensuite, nous gérerons l'implémentation graphique de la solution.

Enfin, si le temps le permet, une foultitude de fonctionnalités supplémentaires peuvent être ajoutées : nouveaux peuples/terrains, séparer la notion de peuple et unité, ajouter des bonus/malus, implémenter un générateur de terrain, une intelligence artificielle...